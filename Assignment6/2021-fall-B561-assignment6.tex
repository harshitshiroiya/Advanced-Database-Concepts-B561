\documentclass{article}

\usepackage{enumitem}
\usepackage{amssymb}
\usepackage{amsmath}
\usepackage{fancyvrb}
\usepackage{venndiagram}
\usepackage{tikz}
\newtheorem{example}{Example}
\usepackage{alltt,xcolor}
\usepackage{graphicx}

\newcommand{\redbullet}{$\textcolor{red}{\bullet}$}
\newcommand{\bluebullet}{$\textcolor{blue}{\bullet}$}

\newcommand{\red}[1]{\textcolor{red}#1}
\newcommand{\green}[1]{{\color{green}#1}}
\newcommand{\blue}[1]{{\color{blue}#1}}
\newcommand{\purple}[1]{{\color{purple}#1}}
\newcommand{\orange}[1]{{\color{orange}#1}}
\newcommand{\hashjoin}[1]{\begin{array}{c} \mathtt{hashJoin} \\ \bowtie\\ {#1}\\ \end{array}}
\newcommand{\mergejoin}[1]{\begin{array}{c} \mathtt{mergeJoin} \\ \bowtie\\ {#1}\\ \end{array}}
\newcommand{\indexjoin}[1]{\begin{array}{c} \mathtt{indexJoin} \\ \bowtie\\ {#1}\\ \end{array}}
\newcommand{\nestedjoin}[1]{\begin{array}{c} \mathtt{nestedLoopJoin} \\ \bowtie\\ {#1}\\ \end{array}}

\newcommand{\sortprojection}[1]{\begin{array}{c} \mathtt{sortProjection} \\ \pi_{#1}\ \end{array}}
\newcommand{\hashprojection}[1]{\begin{array}{c} \mathtt{hashProjection} \\ \pi_{#1}\ \end{array}}
\newcommand{\scan}[1]{\begin{array}{c} \mathtt{scan} \\ {#1}\ \end{array}}
\newcommand{\filter}[1]{\begin{array}{c} \mathtt{filterSelection}\\ \sigma_{#1}\ \end{array}}
\newcommand{\filterscan}[2]{\begin{array}{c} \mathtt{filterScan}\\ \sigma_{#1}(#2)\ \end{array}}
\newcommand{\sortfilterscan}[2]{\begin{array}{c}\mathtt{sort} \mathtt{filterScan}\\ \sigma_{#1}(#2)\ \end{array}}

\begin{document}

\title{B561 Assignment 6\\
Data Generation \\Sorting \\ Indexing \\ (Draft: Version 1)}
\author{Dirk Van Gucht}
\date{}
\maketitle

For this assignment, you will need the material covered in the lectures
\begin{itemize}
\item Lecture 15: External Merge Sorting
\item Lecture 16: Indexing
\item Lecture 17: $\text{B}^+$ trees and Hashing
\end{itemize} 
In addition, you should read the following sections in Chapter 14 and 15 in the textbook \emph{Database Systems The Complete Book} by Garcia-Molina, Ullman, and Widom:
\begin{itemize}
\item Section 14.1: Index-structure Basics
\item Section 14.2: B-Trees
\item Section 14.3: Hash Tables
\item Section 14.7: Bitmap Indexes
\item Section 15.8.1: Multipass Sort-Based Algorithms
\end{itemize} 

In the file {\tt performingExperiments.sql} supplied with this assignment, we have include several
PostgreSQL functions that should be useful for running experiments.     Of course, you may wish to write your own functions and/or adopt the functions in this .sql to suite the required experiments for the various problems in this assignment.

The problems that need to be included in the {\tt assignment6.sql} are marked with a blue bullet \bluebullet.
The problems that need to be included in the {\tt assignment6.pdf} are marked with a red bullet \redbullet.
(You should aim to use Latex to construct your .pdf file.)
In addition, submit a file {\tt assignment6Code.sql} that contains all the sql code that you developed for this assignment.

Practice problems should not be submitted.


\newpage
\section{Data generation}

\paragraph{PostgreSQL functions and clauses}

\medskip
In this assignment there will be a need to do simulations with various-size relations.   Many of these relations will have randomized data.
PostgreSQL provides useful functions and clauses that make such relations:

{\footnotesize
\begin{center}
\begin{tabular}{ll}
$\mathtt{random()}$ & returns a random real number between $0$ and $1$ using the uniform distribution\\
$\mathtt{floor(random() * (u-l+1) + l)::int}$ & returns a random integer in the range $[l,u]$ using the uniform distribution\\
$\mathtt{generate\_series(m,n)}$ & generates the set of integers in the range $[m,n]$\\
$\mathtt{generate\_series(m,n,k)}$ & generates the set of integers in the range $[m,n])$ separated by step-size $k$\\
{\tt order by random()} & randomly orders the tuples that are the result of a query\\
$\mathtt{limit(n)}$ & returns the first $n$ tuples from the result of a query \\ 
$\mathtt{offset(n)}$ & returns all but the first $n$ tuples from the result of a query \\
$\mathtt{limit(m)\ offset(n)}$ & returns the first $m$ tuples from all but the first $n$ tuples from the result of a query\\
$\mathtt{row\_number()}$& is a \emph{window function} that assigns a sequential integer to each row in a result set\\
$\mathtt{vacuum}$ & is a \emph{garbage collection} function to clean and reclaim secondary memory space
\end{tabular}
\end{center}
}
For more detail, consult the manual pages
\begin{alltt}
https://www.postgresql.org/docs/13/functions-math.html   
https://www.postgresql.org/docs/current/functions-srf.html   
https://www.postgresql.org/docs/current/queries-limit.html
https://www.postgresql.org/docs/8.4/functions-window.html
https://www.postgresql.org/docs/9.5/routine-vacuuming.html
\end{alltt}

In the file {\tt performingExperiments.sql} supplied with this assignment, we have include several
PostgreSQL functions that should be useful for running experiments.     Of course, you may wish to write your own functions and/or adopt the functions in this .sql to suite the required experiments for the various problems in this assignment.

\paragraph{Generating sets}
To generate a set, i.e., a unary relation, of $n$ randomly selected integers in the range $[l,u]$, you can use the
following function:\footnote{Typically the
function {\tt SetOfIntegers} returns a bag (multiset) but this is fine for this assignment.  In case we want a
set, we can always eliminate duplicates.}
{\small
\begin{alltt}
\textcolor{blue}{create or replace function SetOfIntegers(n int, l int, u int) 
    returns table (x int) as
$$
    select floor(random() * (u-l+1) + l)::int from generate_series(1,n); 
$$ language sql;}
  \end{alltt}
  }
 \begin{example}
To generate a unary relation with $3$ randomly selected integers in the range $5$ to $10$,  do the following:
{
\begin{center}
\begin{tabular}{l}
\textcolor{blue}{
{\tt select x from SetofIntegers(3,5,10);}}
\end{tabular}
\end{center}
}
Of course, running this query multiple times, result in different sets.
\end{example}

\newpage
\paragraph{Generating binary relations} 
The idea behind generating a set can be generalized to that for the generation of a binary relation.\footnote{Clearly, all of this can be generalized to higher-arity relations.} 
To generate a binary relation of $n$ randomly selected pairs of integers $(x,y)$ such 
$x\in [l_1,u_1]$ and $y\in [l_2,u_2]$, you can use the following function:
{\small
\begin{alltt}
\textcolor{blue}{create or replace function 
BinaryRelationOverIntegers(n int, l_1 int, u_1 int, l_2 int, u_2 int) 
   returns table (x int, y int) as
$$
   select floor(random() * (u_1-l_1+1) + l_1)::int as x, 
          floor(random() * (u_2-l_2+1) + l_2)::int as y
   from   generate_series(1,n);
$$ language sql;}
\end{alltt}
}

\begin{example}
To generate a binary relation with $20$ randomly selected pairs with first components in the range $[3,8]$
and second components in the range $[2,11]$,  do the following:
{
\begin{center}
\begin{tabular}{l}
\textcolor{blue}{
{\tt select x, y from BinaryRelationOverIntegers(20,3,8,2,11);}}
\end{tabular}
\end{center}
}
\end{example}

\paragraph{Generating functions}
A relation generated by {\tt BinaryRelationOverIntegers} is in general not a function since it is
possible that the relation has pairs $(x,y_1)$ and $(x,y_2)$ with
$y_1\neq y_2$.
To create a (partial) \emph{function} $f: [l_1,u_1]\rightarrow [l_2,u_2]$ of $n$ randomly selected
function pairs, we can use the following function:
{\small
\begin{alltt}
\textcolor{blue}{create or replace function 
FunctionOverIntegers(n int, l_1 int, u_1 int, l_2 int, u_2 int) 
   returns table (x int, y int) as
$$
   select x, floor(random() * (u_2-l_2+1) + l_2)::int as y
   from   generate_series(l_1,u_1) x order by random() limit(n);
$$ language sql;}
\end{alltt}
}

\begin{example}
To generate a partial function $[1,20]\rightarrow [3,8]$ of $15$ randomly selection function pairs, do the following:\footnote{When using this function, it is customary to use $n$ such that $n\in [0,u_1-l_1+1]$.}
{
\begin{center}
\begin{tabular}{l}
\textcolor{blue}{
{\tt select x, y from FunctionOverIntegers(15,1,20,3,8);}}
\end{tabular}
\end{center}
}
\end{example}

\newpage
\paragraph{Generating relations with categorical (non-numeric) data}

Thus far, the sets, binary relations, and functions have all just involved integer ranges.  It is possible to include ranges that
have different types including categorical data such as text strings.   
The technique to accomplish this is to first associate with a categorical range an integer range that associate with each element
in the categorical range a unique value of the integer range.
The next example illustrates this.
\begin{example}
Consider the {\tt jobSkill} relation and assume that it contents is
\begin{center}
{\tt
\begin{tabular}{l}
{\tt skill} \\ \hline
\begin{tabular}{l}
AI \\
Databases \\
Networks \\ 
OperatingSystems\\
Programming \\
\end{tabular}
\end{tabular}}
\end{center}
Suppose that we want to generate a {\tt personSkill(pid, skill)} relation.    Let us assume that the {\tt pid}'s are integers in the range
$[1,m]$.  

There are $5$ skills in the {\tt jobSkill} and it is therefore natural
to associate with each skill a separate integer (index value) in the range $[1,5]$.     This can be done with a query involving the
{\tt row\_number()} window function:
\begin{alltt}{\blue{
select skill, row\_number() over (order by skill) as index 
from   Skill;}}
\end{alltt}

The result is the following relation:
\begin{center}
{\tt
\begin{tabular}{lc}
{\tt skill} &{\tt index} \\ \hline
AI & 1\\
Databases & 2\\
Networks & 3\\
OperatingSystems &4\\ 
Programming & 5 \\
\end{tabular}}
\end{center}
Using this technique, we can write a PostgreSQL function  that generates a {\tt personSkill} relation
with $n$ randomly selected $(pid,skill)$ tuples, with {\tt pid}'s in the range $[l,u]$:

{\small
\begin{alltt}
\textcolor{blue}{create or replace function GeneratepersonSkillRelation(n int, l int, u int) 
returns table (pid int, skill text) as
$$
with skillNumber(skill, index) as (select skill, row_number() over (order by skill) 
                                   from   Skill),
     pS as (select x, y
            from   BinaryRelationOverIntegers(n,l,u,1, (select count(1) from Skill)::int))
select x as pid, skill
from   pS join skillNumber on y = index
group by (x, skill) order by 1,2;
$$ language sql;}
\end{alltt}
}
In this function, the {\tt skillNumber} view associates with each job skill an integer index in the range $[1,|{\tt Skill}|]$.
The {\tt pS} view is a randomly generated binary relation with $n$ tuples, with {\tt pid}'s in the range $[l,u]$,
and skill numbers in the range $[1,|{\tt Skill}|]$.  The \blue{{\tt join}} operation associates the numeric range with the {\tt Skill}
range. The `\blue{\tt group by (x, skill) order by 1,2}' clause eliminates duplicate tuples and orders the result.

The query 
\begin{alltt}\blue{
select * from GeneratepersonSkillRelation(10,1,15);}
\end{alltt}
may make the {\tt personSkill} relation:

\begin{center}
{\tt
\begin{tabular}{cl}
 pid      & skill       \\ \hline
   1 & AI \\
   2 & Programming\\
   3 & Databases\\
   4 & Databases\\
   6 & Networks\\
   6 & OperatingSystems\\
   6 & Programming\\
   9 & Databases\\
  14 & Databases\\
  14 & Networks\\
\end{tabular}}
\end{center}  

\end{example}


\paragraph{Problems}
We now turn to the problems in this section.
\begin{enumerate}[resume]
\item\label{probdistribution}\bluebullet\  Given a discrete probability mass function $P$, as specified in a relation {\tt P(outcome: int, probability: float)}, over a range of possible outcomes $[u_2,l_2]$, design a PostgreSQL function 
\[\mathtt{RelationOverProbabilityFunction}(n,l_1,u_1,l_2,u_2)\] that generates a relation of up to $n$ pairs $(x,y)$ such that
\begin{itemize}
\item  $x$ is uniformly selected in the range $[l_1,u_1]$; and
\item  $y$ is selected in accordance with the probability mass function $P$ in the range $[l_2,u_2]$.
\end{itemize}

An example of a possible $P$ as stored in relation {\tt P} is as follows:\footnote{Notice that the sum of the probabilities in the {\tt probability} column in {\tt P} sum to $1$.}
\begin{center}
{\tt 
\begin{tabular}{c} \\
P \\
\begin{tabular}{cc}
outcome & probability \\ \hline
1 & $0.25$ \\ 
2 & $0.10$ \\
3 & $0.40$ \\
4 & $0.10$ \\
5 & $0.15$ \\
\end{tabular}
\end{tabular}
}
\end{center}

Note that when $P$ is the uniform probability mass function, then
\[\mathtt{RelationOverProbabilityFunction}\]
and
\[\mathtt{BinaryRelationOverIntegers}\] are the same binary relation producing functions.

{\bf Hint}:  For insight into this problem, consult the method of 
\emph{Inverse Transform Sampling} for discrete probability mass functions.

Test your function for the cases listed in the {\tt Assignment-Script-2021-Fall-assignment6.sql} file.

\item \textcolor{red}{\bf Practice Problem-not graded}.

Use the technique in Problem~\ref{probdistribution} and the method for generating categorical data discussed above to
write a PostgreSQL function that generates a {\tt personSkill} relation, given a probability mass function over the {\tt Skill} relation.   

Your function should work for any {\tt Skill} relation and any probability distribution defined over it.
\end{enumerate}

Test your function for the cases listed in the {\tt Assignment-Script-2021-Fall-assignment6.sql} file.

\newpage
\section{Sorting}

We have learned about \emph{external sorting}.   The problems in this section are designed to look into this sorting method as it implemented in PostgreSQL.

\begin{enumerate}[resume]
\item \redbullet\ \label{sortTime}   Create successively larger sets of $n$ randomly selected integers in the range $[1,n]$.     You can do this  using the following function and with the following experiment.\footnote{You should make it a habit to use the PostgreSQL \blue{\tt vacuum} function to perform garbage collection between experiments.} 

{\small \begin{alltt}
\textcolor{blue}{create or replace function makeS (n integer)
returns void as
$$
begin 
    drop table if exists S;
    create table S (x int);
    insert into S select * from SetOfIntegers(n,1,n);
end;    
$$ language plpgsql;}
\end{alltt}}
This function generates a bag $S$ of size $n$, with randomly select integers in the range $[1,n]$.
Now consider the following SQL statements:
\begin{center}
\begin{alltt}
\textcolor{blue}{select makeS(10);
explain analyze {select x from S;}
explain analyze {select x from S order by 1;}}
\end{alltt}
\end{center}

\begin{itemize}
\item The `\blue{\tt select makeS(10)}' statement makes a bag $\mathtt{S}$ with 10 elements;
\item The `\blue{\tt explain analyze {select x from S}}' statement provides the query plan and execution time in milliseconds (ms) for a simple scan of {\tt S};
\item The `\blue{\tt explain analyze {select x from S order by 1}}' statement provides the query plan and \textcolor{red}{execution time} in milliseconds (ms) for sorting {\tt S}.\footnote{Recall that 1ms is $\frac{1}{1000}$ second.}
\end{itemize}

\begin{center}
{\footnotesize
\begin{alltt}
\textcolor{blue}{
    QUERY PLAN                                              
------------------------------------------------------------------------------------------------------
 Sort  (cost=179.78..186.16 rows=2550 width=4) (actual time=0.025..0.026 rows=10 loops=1)
   Sort Key: x
   Sort Method: quicksort  Memory: 25kB
   ->  Seq Scan on s  (cost=0.00..35.50 rows=2550 width=4) (actual time=0.004..0.005 rows=10 loops=1)
 Planning Time: 0.069 ms
 \textcolor{red}{Execution Time: 0.034 ms}}
\end{alltt}
}
\end{center}

Now construct the following timing table:\footnote{It is possible that you may not be able to run the experiments for the largest {\tt S}.   If that is the case, just report the results for the smaller sizes.}
\begin{center}
\begin{tabular}{c|c|c}
size $n$ of relation {\tt S} & avg execution time to \textcolor{red}{scan} {\tt S} (in ms) &avg execution time to \textcolor{red}{sort} {\tt S} (in ms) \\ \hline
$10^1$ & & \\
$10^2$ & & \\
$10^3$ &  & \\
$10^4$ & & \\
$10^5$ &  & \\
$10^6$ &  & \\
$10^7$ & & \\
$10^8$ &  & \\
\end{tabular}
\end{center}

\begin{enumerate}
\item\label{sortTimea} What are your observations about the query plans for the scanning and sorting of such differently sized bags {\tt S}?
\item What do you observe about the execution time to sort {\tt S} as a function of $n$?
\item Does this conform with the formal time complexity of (external) sorting?  Explain.
\item It is possible to set the \emph{working memory} of PostgreSQL using the \blue{{\tt set work\_mem}}
command.   For example, \blue{{\tt set work\_mem = '16MB'}} sets the working memory to 16MB.\footnote{The default working memory size is 4MB.}   The smallest possible working memory in postgreSQL is 64kB and the largest depends on you computer memory size.   But you can try for example 1GB.

Repeat question~\ref{sortTimea} for memory sizes 64kB and 1GB and report your observations.

\item  Now create a relation {\tt indexedS(x integer)} and create a Btree index on {\tt indexedS} and insert into {\tt indexedS} the sorted relation {\tt S}.\footnote{For information about \emph{indexes} in PostgreSQL consult the manual page
{\tt https://www.postgresql.org/docs/13/indexes.html}.}
\begin{alltt}
\textcolor{blue}{
create table indexedS (x integer);
create index on indexedS using btree (x);
insert into indexedS select x from S order by 1;}
\end{alltt}
Then run the range query 
\begin{alltt}
\textcolor{blue}{select x from indexedS where x between 1 and \textcolor{red}{n};}
\end{alltt}
where \textcolor{red}{\tt n} denotes the size of {\tt S}.  

Then construct the following table which contains (a) the average execution time to build the btree index
and (2) the average time to run the range query.
\begin{center}
{\footnotesize
\begin{tabular}{c|c|c}
size $n$ of relation {\tt S} & avg execution time to create index {\tt indexedS} & avg execution time for range query (in ms) \\ \hline
$10^1$ & &\\
$10^2$ & &\\
$10^3$ &  &\\
$10^4$ & &\\
$10^5$ &  &\\
$10^6$ &  &\\
$10^7$ & &\\
$10^8$ & &\\
\end{tabular}
}
\end{center}
What are your observations about the query plans and execution times to create  {\tt indexedS} and the execution times for sorting the differently sized bags {\tt indexedS}?   Compare your answer with those for the above sorting problems.
\end{enumerate}
\item \textcolor{red}{\bf Practice problem-not graded}.

Typically, the {\tt makeS} function returns a bag instead of a set.   In the problems in this section, you are to conduct
experiments to measure the execution times to eliminate duplicates.   

\begin{enumerate}
\item\label{duplicateDistinct} 
Write a SQL query that uses the \blue{\tt DISTINCT} clause to eliminate duplicates in {\tt S} and 
report you results in a table such as that in Problem~\ref{sortTimea}.
\item\label{duplicateGroupBy}  Write a SQL query that uses the \blue{\tt GROUP BY} clause to eliminate duplicates in {\tt S} and 
report you results in a table such as that in Problem~\ref{sortTimea}.
\item Compare and contrast the results you have obtained in problems~\ref{duplicateDistinct} and \ref{duplicateGroupBy}.
Again, consider using \blue{\tt explain analyze} to look at query plans.

\end{enumerate}

\end{enumerate}

\newpage
\section{Indexes and Indexing}
Indexes on data (1) permit faster lookup on data items and (2) may speed up query processing on such data.
Speedups can be substantial.     The purpose of the problems in this section are to explore this.   Some other problems in this section
are designed to understand the workings of the \emph{B$^+$-tree} and the \emph{extensible hashing} data structures.

\paragraph{Discussion}
PostgreSQL permit the creation of a variety of indexes on tables.   We will review such {\tt index creation} and examine their
impact on data lookup and query processing.
For more details, see the PostgreSQL manual:
\begin{center}
\textcolor{blue}{https://www.postgresql.org/docs/13/indexes.html}
\end{center}

\begin{example}
The following SQL statements create indexes on columns or combinations
of columns of the {\tt personSkill} relation.\footnote{Incidentally, when a primary key is declared when 
a relation is created, PostgreSQL will create a btree index on this key for the relation.}   Notice that there are \[2^{arity(\mathtt{personSkill})}-1 = 2^2-1 = 3\] such possible
indexes.  

{\small
\begin{alltt}
\textcolor{blue}{
create index pid_index on personSkill (pid);                  \textcolor{red}{-- index on pid attribute}
create index skill_index on personSkill (skill);              \textcolor{red}{-- index on skill attribute}
create index pid_skill_index on personSkill (pid,skill);      \textcolor{red}{-- index (pid, skill)}
}
\end{alltt}}
\end{example}
\begin{example}
It is possible to declare the type of index: {\tt btree} or {\tt hash}. When no index type is specified, the default is {\tt btree}.
If instead of a Btree, a \emph{hash index} is desired, then it is necessary
to specify a \textcolor{teal}{\tt using hash} qualifier:
{\small
\begin{alltt}
\textcolor{blue}{
create index pid_hash on personSkill \textcolor{teal}{using hash} (pid);  \textcolor{red}{-- hash index on pid attribute }                                                                                                                                              
}
\end{alltt}}
\end{example}

\begin{example}
It is possible to create an index on a relation based on a \emph{scalar expression} or a \emph{function} defined over the attributes
of that relation.    Consider the following (immutable) function which computes the number of skills of a person:
{\small
\begin{alltt}
\textcolor{blue}{
create or replace function numberOfSkills(p integer) returns integer as
$$
    select count(1)::int 
    from   personSkill 
    where  pid = p;
$$ language SQL immutable;}
\end{alltt}
}                                                                                                  
Then the following is an index defined on the \textcolor{teal}{\tt numberOfSkills} values of persons:
{\small
\begin{alltt}
\textcolor{blue}{
create index numberOfSkills_index on personSkill (\textcolor{teal}{numberOfSkills(pid)});}
\end{alltt}
Such an index is useful for queries that use this function such as
\begin{alltt}
\textcolor{blue}{select pid, skill from personSkill where \textcolor{teal}{numberOfSkills(pid)} > 2;}
\end{alltt}
}
\end{example}

We now turn to the problems in this section.

\begin{enumerate}[resume]
\item\label{secondaryIndexes}\bluebullet\ Consider a relation {\tt Student(sid text, sname text, major, year)}.   A tuple
$(s, n, m, y)$ is in {\tt Student} when $s$ is the sid of a student and $n$, $m$, and $y$ are that
student's name, major, and birth year.  Further,
consider a relation {\tt Enroll(sid text, cno text, grade text)}.   A triple $(s, c, g)$ is in {\tt Enroll} when the student with sid $s$ was enrolled in the course with cname $c$ and obtained a letter grade $g$ in that course. 

We are interested in answering queries of the form
{\footnotesize
\begin{alltt}{\blue{
select sid, sname, major, byear 
from   Student 
where  sid in (select sid
               from   Enroll sid
               where  cno = \red{c} [and|or|and not] grade = \red{g});}}
\end{alltt}  
}
Here \textcolor{red}{\tt c} denotes a course name and \textcolor{red}{\tt g} denotes a letter grade.

Read Section 14.1.7 `Indirection in Secondary Indexes' in your textbook \emph{Database Systems The Complete Book} by Garcia-Molina, Ullman, and Widom.   Of particular interest are (a) the concept of \emph{buckets} (Figure 14.7) which are sets of tids and (b)
the technique of performing set operations (like intersections) on relevant buckets (Figure 14.8) to answer queries of the form as shown above.

The goal of this problem is to use object-relational SQL to simulate these concepts. To make things more concrete, consider the following {\tt Student} and {\tt Enroll} relations:


\begin{center}
{\footnotesize
\begin{alltt}
Student:
 sid  | sname  |  major  | byear 
------+--------+---------+-------
 s100 | Eric   | CS      |  1988
 s101 | Nick   | Math    |  1991
 s102 | Chris  | Biology |  1977
 s103 | Dinska | CS      |  1978
 s104 | Zanna  | Math    |  2001
 s105 | Vince  | CS      |  2001
\end{alltt}
}
\end{center}

\begin{center}
{\footnotesize
\begin{alltt}
Enroll:
 sid  | cno  | grade 
------+------+-------
 s100 | c200 | A
 s100 | c201 | B
 s100 | c202 | A
 s101 | c200 | B
 s101 | c201 | A
 s101 | c202 | A
 s101 | c301 | C
 s101 | c302 | A
 s102 | c200 | B
 s102 | c202 | A
 s102 | c301 | B
 s102 | c302 | A
 s103 | c201 | A
 s104 | c201 | D
\end{alltt}
}
\end{center}

Now consider associating a tuple id (tid) with each tuple in {\tt Enroll}:
{\footnotesize
\begin{alltt}
 tid | sid  | cno  | grade 
-----+------+------+-------
   1 | s100 | c200 | A
   2 | s100 | c201 | B
   3 | s100 | c202 | A
   4 | s101 | c200 | B
   5 | s101 | c201 | A
   6 | s101 | c202 | A
   7 | s101 | c301 | C
   8 | s101 | c302 | A
   9 | s102 | c200 | B
  10 | s102 | c202 | A
  11 | s102 | c301 | B
  12 | s102 | c302 | A
  13 | s103 | c201 | A
  14 | s104 | c201 | D
\end{alltt}
}

Use object-relational SQL to construct two secondary indexes {\tt indexOnCno} and {\tt indexOnGrade} on the {\tt Enroll} relation.  These indexes should be stored in two separate relations which you can conveniently call
{\tt indexOnCno} and {\tt indexOnGrade}, respectively.
two object-relational views in a manner that simulates the situation illustrated in Figure 14.8.  
In particular, do {\bf not} use the `\blue{\tt create index}' mechanism of SQL to construct these indexes.

Then, using the {\tt indexOnCno} and {\tt indexOnGrade} views and the technique of \emph{intersecting buckets}, write a function 
{\tt FindStudents(booleanOperation text, cno text, grade text)} that can be used to answer queries of the form as shown above.   (Here the booleanOperation is a string which can be 'and', 'or', or 'and not'.)

For example, the query
{\footnotesize
\begin{alltt}
\blue{select * from FindStudents('and', 'c202', 'A');}
\end{alltt}}
should return the same result as that of the query
{\footnotesize
\begin{alltt}{\blue{
select sid, sname, major, byear 
from   Student 
where  sid in (select sid
               from   Enroll sid
               where  cno = 'c202' and grade = 'A');}}
\end{alltt}  
}

Test your solution for the following cases on the {\tt Student} and {\tt Enroll} relation given for this problem:
\begin{enumerate}
\item {\tt select * from FindStudents('and', 'c202', 'A');}
\item {\tt select * from FindStudents('or', 'c202', 'A');}
\item {\tt select * from FindStudents('and not', 'c202', 'A');}
\end{enumerate}

\item \textcolor{red}{\bf Practice problem--not graded}.
Read Section~14.7 `Bitmap Indexes' in your textbook \emph{Database Systems The Complete Book} by Garcia-Molina, Ullman, and Widom.   In particular, look at Example~14.39 for an example of a bitmap index for a secondary index.

Next, revisit Problem~\ref{secondaryIndexes}.   There, we considered two secondary indexes {\tt indexOnCno} and
{\tt indexOnGrade}.  We will now consider the corresponding bitmap indexes
{\tt bitmapIndexOnCno} and {\tt bitmapIndexOnGrade}:

{\footnotesize
\begin{alltt}
bitmapIndexOnCno
 cno  |     bit-vector     
------+----------------
 c200 | 10010000100000
 c201 | 01001000000011
 c202 | 00100100010000
 c301 | 00000010001000
 c302 | 00000001000100
\end{alltt}}
and
{\footnotesize
\begin{alltt}
bitmapIndexOnGrade
 grade |     bit-vector     
-------+----------------
 A     | 10101101010110
 B     | 01010000101000
 C     | 00000010000000
 D     | 00000000000001
\end{alltt}}

Use object-relational SQL to construct two secondary indexes {\tt bitmapIndexOnCno} and {\tt bitmapIndexOnGrade} as
two object-relational relations in a manner that simulates the situation just illustrated above.

Then, using the {\tt bitmapIndexOnCno} and {\tt bitmapIndexOnGrade} relations and the technique of forming the bitmap-{\tt AND}, bitmap-{\tt OR}, and bitmap-{\tt AND NOT} of two bit-vectors,  write a function 
{\tt FindStudents(booleanOperation text, cno text, grade text)} that can be used to answer queries of the form as shown in Problem~\ref{secondaryIndexes}.

For example, the query
{\footnotesize
\begin{alltt}
\blue{select * from FindStudents('and', 'c202', 'A');}
\end{alltt}}
should return the same result as that of the query
{\footnotesize
\begin{alltt}{\blue{
select sid, sname, major, byear 
from   Student 
where  sid in (select sid
               from   Enroll sid
               where  cno = 'c202' and grade = 'A');}}
\end{alltt}  
}


Test your solution for the following cases on the {\tt Student} and {\tt Enroll} relation given for this problem:
\begin{enumerate}
\item {\tt select * from FindStudents('and', 'c202', 'A');}
\item {\tt select * from FindStudents('or', 'c202', 'A');}
\item {\tt select * from FindStudents('and not', 'c202', 'A');}
\end{enumerate}
\end{enumerate}

\begin{enumerate}[resume]
\item \redbullet\ 
Consider the following parameters:
\begin{center}
\begin{tabular}{lcl}
block size         & = & 4096 bytes \\
block-address size & = & 9 bytes \\
block access time (I/O operation) & = & 10 ms (micro seconds) \\
record size       & = & 150 bytes \\
record primary key size   & = & 8 bytes \\
\end{tabular}
\end{center}
Assume that there is a B$^+$-tree, adhering to these parameters, that
indexes 1 billion ($10^9$) records on their primary key values.

Provide answers to the following problems and 
show the intermediate computations leading to your answers.

\begin{enumerate}
\item   Specify (in ms) the minimum time to determine if there is a record with
  key $k$ in the B$^+$-tree.
(In this problem, you can not assume that a key value that appears in an non-leaf node has a corresponding record with that key value.)

\item \label{maximum} Specify (in ms) the maximum time to insert a record
with key $k$ in the B$^+$-tree assuming that this record was not already in the data file.

\item  How large must main memory be to hold the first two levels of the B$^+$-tree? How about the first three levels?
\end{enumerate}

\newpage
\item \redbullet\  Consider the following B$^+$-tree of order 2 that holds records
  with keys 2, 8, and 11.\footnote{Recall that (a) an internal node of a
  B$^+$-tree of order 2 can have either 1 or 2 keys values, and 2 or 3
  sub-trees, and (b) a leaf node can have either 1 or 2 key values.}
  
\begin{figure}[h]
\centering
\includegraphics[height = 1.5in, width = 3in]{btree.png}
\end{figure}
\begin{enumerate}

\item \label{btree-insert} Show the contents of your $B^+$-tree after
inserting records with keys 4, 10, 12, 1, 7, and 5 in that order.

{\bf Strategy for splitting leaf nodes}: when a leaf node $n$ needs to be split into two nodes $n_1$ and $n_2$ (where $n_1$ will point to $n_2$), then use the
rule that an even number of keys in $n$ is moved into $n_1$ and an odd number of keys is moved into $n_2$.   So if $n$ becomes conceptually \fbox{$k_1|k_2|k_3$} then $n_1$ should be \fbox{$k_1|k_2$} and $n_2$ should be \fbox{$k_3|\ \ $} and $n_1 \rightarrow n_2$.

\item\label{btreedelete} Starting from your answer in question~\ref{btree-insert}, show
  the contents of your $B^+$-tree after deleting records with keys 12, 2, and 11, in that order.
\item Starting from your answer in question~\ref{btreedelete}, show
  the contents of your $B^+$-tree after deleting records with keys 5, 1, and 4, in that order.
\end{enumerate}

\newpage
\item \redbullet\ 
Consider an extensible hashing data structure wherein (1) the
initial global depth is set at $1$ and (2) all directory pointers point
to the same {\bf empty} bucket which has local depth $0$. So the hashing
structure looks like this:
\begin{figure}[h]
        \centering
        \includegraphics[height = 1.5in, width = 3in]{hash.png}
    \end{figure}
\bigskip
Assume that a bucket can hold at most two records.

\begin{enumerate}
\item\label{ext-insert} Show the state of the hash data structure after each of the
  following insert sequences:\footnote{You should interpret the key values as bit strings of length 4.   So for example,  key value 7 is represented as the bit string 0111 
and key value 2 is represented as the bit string 0010.}

\begin{enumerate}
\item records with keys 6 and 7.
\item records with keys 1 and 2.
\item records with keys 9 and 4.
\item records with keys 8 and 3.
\end{enumerate}


\item \label{ext-two} Starting from the answer you obtained for
  Question~\ref{ext-insert}, show the state of the hash data structure
  after each of the following delete sequences:

\begin{enumerate}
\item records with keys 9 and 6.
\item records with keys 4 and 1.
\item records with keys 2 and 8.
\end{enumerate}

As in the text book, the bit strings are interpreted starting from their left-most bit and continuing to the right subsequent bits. 

\end{enumerate}
\end{enumerate}



\newpage
The goal in the next problems is to study the behavior of indexes for various different sized instances\footnote{(Three different sizes, small, medium, large suffice for your experiment.} of the {\tt Person}, {\tt worksFor}, and {\tt Knows} relations
and for various queries:
\begin{enumerate}[resume]
\item \redbullet\  Create an appropriate index on the {\tt worksFor} relation that speedups the lookup query
\begin{alltt}\textcolor{blue}{
select pid from worksFor where cname = \red{c};}
\end{alltt}
Here \textcolor{red}{c} is a company name.

Illustrate this speedup by finding the execution times for this query for various sizes of the {\tt worksFor} relation.  
\item \redbullet\  Create an appropriate index on the {\tt worksFor} relation that speedups the range query
\begin{alltt}\textcolor{blue}{
select pid, cname
from   worksFor
where  \textcolor{red}{s1} <= salary and salary <= \textcolor{red}{s2};}
\end{alltt}
Here \textcolor{red}{s1} and \textcolor{red}{s2} are two salaries with $\text{s1} < \text{s2}$.

Illustrate this speedup by finding the execution times for this query for various sizes of the {\tt worksFor} relation.
\item \redbullet\  Create indexes on the {\tt worksFor} relation that speedup the multiple conditions query
\begin{alltt}\textcolor{blue}{
select pid
from   worksFor
where  salary = \red{s} and cname = \red{c};}
\end{alltt}

Here \textcolor{red}{s} is a salary and \textcolor{red}{c} is a company name.

Illustrate this speedup by finding the execution times for this query for various sizes of the {\tt worksFor} relation.
\item \redbullet\  Create indexes on the appropriate relations that speedup the semi-join [anti semi-join] query
\begin{alltt}\textcolor{blue}{
select pid, pname 
from   Person 
where  pid [not] in (select pid from worksFor where cname = \red{c});}
\end{alltt}

Here \textcolor{red}{c} is a company name.

Illustrate this speedup by finding the execution times for this query for various sizes of the {\tt Person} and {\tt worksFor} relations.

\item \textcolor{red}{\bf Practice problem--not graded.}

Create indexes that speedup the path-discovery query
\begin{alltt}\textcolor{blue}{
select distinct k1.pid1, k3.pid2
from   knows k1, knows k2, knows k3 
where  k1.pid2 = k2.pid1 and k2.pid2 = k3.pid1;}
\end{alltt}
Illustrate this speedup by finding the execution times for this query for various sizes of the {\tt Knows} relation.
\end{enumerate}

\end{document}
