\documentclass{article}
\usepackage{enumitem}
\newtheorem{example}{Example}
\begin{document}

\title{Fall B561 Assignment 1 \\
Relational Databases, Expressing Queries and Constraints in SQL and in Tuple Relational Calculus (TRC)\footnote{This assignment covers lectures 1 through 4}.}
\author{Dirk Van Gucht}
\date{Released: Tuesday August 24, 2021\\ Due: Wednesday September 8, 2021 by 11:45pm}
\maketitle


\section{Introduction}

The goals for this assignment are to 
\begin{enumerate}
\item become familiar with the PostgreSQL system\footnote{To solve this assignment, you will need to download and install PostgreSQL (version 12 or higher) on your computer.};
\item create a relational database and populate it with data;
\item examine the side-effects on the state of the database caused by inserts and deletes in the presence or absence
of primary and foreign key constraints;
\item formulate some queries in SQL and evaluate them in PostgreSQL; and
\item translate TRC queries to SQL and 
formulate queries and constraints in TRC.\footnote{To solve problems related to TRC, follow the syntax and semantics described in the {\tt TRC\_SQL.pdf} document
in the module \emph{Tuple Relational Calculus and SQL (lecture 4)}.   That document contains multiple examples of
TRC queries and constraints and how they can be translated to SQL.}
\end{enumerate}

To turn in your assignment, you will need to upload to Canvas a single file with name {\tt assignment1.sql} which contains 
the necessary SQL statements that solve the problems in this assignment.   
The {\tt assignment1.sql} file must be such that the AI's can run it in their PostgreSQL environment.  
In addition, you will need to upload a separate {\tt assignment1.txt} file that contains the results of running
your queries.
We have posted the exact requirements and an example for uploading your solution files.  (See the module
{\tt Instructions for turning in assignments}.)
Finally, you will need to upload an {\tt assignment1.pdf} file that contains the solutions for problems related to TRC.\footnote{It is strongly recommended that you use Latex to write TRC formulas and queries.
For a good way to learn about Latex, look at 
%\begin{verbatim}
https://www.overleaf.com/learn/latex/Free\_online\_introduction\_to\_LaTeX\_(part\_1).
%\end{verbatim}
You can also inspect the Latex source code for this assignment as well as the document TRC\_SQL.tex provided in Module 4.}



For the problems in this assignment we will use the following database schema:\footnote{The primary key, which may consist of one or more attributes, of each of these relations is underlined.}

\begin{center}
{\tt
  \begin{tabular}{l}
  {Person}($\underline{\tt pid}$, pname, city) \\
  {Company}($\underline{\tt cname}, {\tt headquarter}$) \\
  {Skill}($\underline{\tt skill}$) \\
  {worksFor}($\underline{\tt pid}$, cname, salary) \\
  {companyLocation}($\underline{\tt cname}, {\tt city}$) \\
  {personSkill}($\underline{\tt pid, skill}$) \\
  {hasManager}($\underline{\tt eid, mid}$) \\
  \end{tabular}
  }
\end{center}

In this database we maintain a set of persons ({\tt Person}), a set
of companies ({\tt Company}), and a set of (job) skills ({\tt Skill}).  
The {\tt pname} attribute in {\tt Person} is the name of the person.  
The {\tt city} attribute in {\tt Person} specifies the city in which the person lives.  
The {\tt cname} attribute in {\tt Company} is the name of the company.
The {\tt headquarter} attribute in {\tt Company} is the name of the city wherein the company has its headquarter.
The {\tt skill} attribute in {\tt Skill} is the name of a (job) skill.

A person can work for at most one company. This information is maintained in the {\tt worksFor} relation. (We permit that a person does not work for any company.)
The {\tt salary} attribute in {\tt worksFor} specifies the salary made by the person.

The {\tt city} attribute in {\tt companyLocation} indicates a city in which the company is located.
(Companies may be located in multiple cities.)

A person can have multiple job skills. This information is maintained in the {\tt personSkill} relation.  A job skill can be
the job skill of multiple persons.  (A person may not have any job skills, and a job skill may
have no persons with that skill.)

A pair $(e,m)$ in {\tt hasManager} indicates that person $e$ has  
person $m$ as one of his or her managers.
We permit that an employee has multiple managers and that a manager  may manage
multiple employees.  (It is possible that an employee has no manager
and that an employee is not a manager.)
We further require that 
an employee and his or her managers must work for the
same company.

The domain for the attributes {\tt pid}, {\tt salary}, {\tt eid}, and {\tt mid} is {\tt integer}.   The domain for all other attributes is {\tt text}.
\newpage
We assume the following foreign key constraints:
\begin{itemize}
\item {\tt pid} is a foreign key in {\tt worksFor} referencing the primary key {\tt pid} in {\tt Person};
\item {\tt cname} is a foreign key in {\tt worksFor} referencing the primary key {\tt cname} in {\tt Company};
\item {\tt cname} is a foreign key in {\tt companyLocation} referencing the primary key {\tt cname} in {\tt Company};
\item {\tt pid} is a foreign key in {\tt personSkill} referencing the primary key {\tt pid} in {\tt Person};
\item {\tt skill} is a foreign key in {\tt personSkill} referencing the primary key {\tt skill} in {\tt Skill};
\item {\tt eid} is a foreign key in {\tt hasManager} referencing the primary key {\tt pid} in {\tt Person}; and
\item {\tt mid} is a foreign key in {\tt hasManager} referencing the primary key {\tt pid} in {\tt Person};

\end{itemize}

The file {\tt data.sql} contains the data supplied for this assignment.

\newpage
\section{Database creation and impact of constraints on insert and delete statements.}

Create a database in PostgreSQL that stores the data provided in the {\tt data.sql} file.
Make sure to specify primary and foreign keys.

\begin{enumerate}
\item Provide 4 conceptually different examples that illustrate how the presence or absence of primary and
  foreign keys affect insert and deletes in these relations.   To solve this problem,
  you will need to experiment with the  relation schemas and instances for this assignment.   
  For example, 
  you should consider altering primary keys and foreign key constraints and
  then consider various sequences of insert and delete operations.   You may need to change some of the
  relation instances to observe the desired effects.
  Certain inserts and deletes should succeed but other should generate error conditions.
  (Consider the lecture notes about keys, foreign keys, and inserts and deletes as a guide to solve this problem.)
  \end{enumerate}
 \newpage
  \section{Formulating queries in SQL}\label{SQLQueries}

For this assignment,  you are required to 
use  tuple variables in your SQL statements.   
For example, in formulating the query ``Find the pid and pname of each person who lives in Bloomington" you should write the query

\begin{center}{\tt
\begin{tabular}{ll}
SELECT &p.pid, p.pname\\
FROM & Person p\\
WHERE & p.city = `Bloomington'
\end{tabular}}
\end{center}
rather than
\begin{center}{\tt
\begin{tabular}{ll}
SELECT &pid, pname\\
FROM & Person\\
WHERE & city = `Bloomington' 
\end{tabular}}
\end{center}

Write SQL statements for the following queries.   
Make sure that each of your queries returns a set but not a bag.  In other words, make appropriate use of the
{\tt DISTINCT} clause where necessary.

 You can {\bf not} use the SQL JOIN operations or SQL aggregate functions
such as {\tt COUNT}, {\tt SUM}, {\tt MAX}, {\tt MIN}, etc in your solutions.
\begin{enumerate}[resume]
\item Find the pid, pname of each person who (a) lives in Bloomington, 
(b) works for a company where he or she earn a salary that is higher than 30000, and (c) has at least one manager.
\item Find the pairs $(c_1, c_2)$ of names of companies whose headquarters are located in the same city.
\item Find the pid and pname of each person who lives in a city that is different than each city in which his or her managers live.
(Persons who have no manager should not be included in the answer.)
\item Find each skill that is the skill of at most 2 persons.
\item Find the pid, pname, and salary of each employee who has at least two managers such that these managers have a common job skill but provided that it is not the `Networks' skill.
\item Find the cname of each company that not only employs persons                                                                      
who live in MountainView. 
(In other words, there exists at least one employee of such a company
who does not live in MountainView.)
\item For each company, list its name along with the highest salary made by employees who work for it.
\item Find the pid and pname of each employee who has a salary that is higher than the salary of each of his or her managers.
(Employees who have no manager should not be included.)

\end{enumerate}

\section{Translating TRC queries to SQL}

Consider the following queries formulated in TRC.   Translate each of these queries to an equivalent SQL query.\footnote{
You can not use SQL {\tt JOIN} operations or
aggregate functions.}

You should note that this translating, modulo the handling of universal quantifiers, is almost a syntactic rewrite of the way in which the queries are formulated in TRC.
This underscores the close correspondence between TRC and SQL.

The SQL queries should be included in the {\tt assignment1.sql} file and their outputs should be reported in the
{\tt assignment.txt} file.

\begin{enumerate}[resume]
\item 
{\small
\[
\begin{array}{ll}
\{p.pid, p.pname, w.cname, w.salary\mid Person(p) \land worksFor(w)\land p.pid = w.pid\\
\qquad p.city = \mbox{\rm `Bloomington'}\, \land\, 40000 \leq w.salary\, \land\, w.cname \neq \mbox{\rm `Apple'}\}.
\end{array}
\]
}


\item 
{\small
\[
\begin{array}{ll}
\{p.pid, p.pname\mid Person(p) \land \\
\quad \exists c \exists w (Company(c) \land worksFor(w) \land c.cname = w.cname \land p.pid = w.pid \land c.headquarter = \mbox{\rm `LosGatos'} \land \\
\quad \quad \exists hm \exists m (hasManager(hm) \land Person(m)\land 
 hm.eid = p.pid \land hm.mid = m.pid \land m.city \neq \mbox{\rm `LosGatos}))\}.
\end{array}
\]
}

In abbreviated form,
{\small
\[
\begin{array}{ll}
\{p.pid, p.pname\mid Person(p) \land \\
\quad \exists c\in Company\, \exists w\in worksFor (c.cname = w.cname \land p.pid = w.pid \land c.headquarter = \mbox{\rm `LosGatos'} \land \\
\quad \quad \exists hm\in hasManager\, \exists m\in Person (hm.eid = p.pid \land hm.mid = m.pid \land m.city \neq \mbox{\rm `LosGatos}))\}.
\end{array}
\]
}

\item 
{\small
\[
\begin{array}{ll}
\{s.skill\mid Skill(s) \land \lnot (\exists p\exists ps\, Person(p) \land personSkill(ps)\land  p.pid = ps.pid\land \\
\qquad ps.skill = s.skill \land p.city = \mbox{`Bloomington'})\}.
\end{array}
\]
}

In abbreviated form,
{\small
\[
\begin{array}{ll}
\{s.skill\mid Skill(s) \land \lnot (\exists p\in Person\, \exists ps\in personSkill (p.pid = ps.pid\land \\
\qquad ps.skill = s.skill \land p.city = \mbox{`Bloomington'})\}.
\end{array}
\]
}

\item 
{\small
\[
\begin{array}{ll}
\{m.pid, m.pname\mid Person(m) \land \\
\qquad \forall hm ((hasManager(hm)\land hm.mid = m.pid)\rightarrow 
\exists e (Person(e)\land hm.eid = e.pid \land e.city = m.city))\}
\end{array}
\]
}

In abbreviated form,
{\small
\[
\begin{array}{ll}
\{m.pid, m.pname\mid Person(m) \land \\
\qquad \forall hm\in hasManager(hm.mid = m.pid \rightarrow 
\exists e\in Person (hm.eid = e.pid \land e.city = m.city))\}
\end{array}
\]
}

\end{enumerate}
\newpage
\section{Formulating queries in the Tuple Relational Calculus}

Formulate each of the queries in the even-numbered  problems (i.e., problems 2, 4, 6, and 8) 
in Section~\ref{SQLQueries} as TRC queries.

The solutions of these problems should be included in the {\tt assignment1.pdf} file.

\begin{enumerate}[resume]
\item (Problem 2) Find the pid, pname of each person who (a) lives in Bloomington, 
(b) works for a company where he or she earn a salary that is higher than 30000, and (c) has at least one manager.

\item (Problem 4) Find the pid and pname of each person who lives in a city that is different than each city in which his or her managers live.
(Persons who have no manager should not be included in the answer.)

\item (Problem 6) Find the pid, pname, and salary of each employee who has at least two managers such that these managers have a common job skill but provided that it is not the `Networks' skill.

\item (Problem 8) For each company, list its name along with the highest salary made by employees who work for it.

\end{enumerate}
\newpage

\section{Formulating constraints in the Tuple Relational Calculus}

Formulate the following constraints in TRC and as boolean SQL queries.    

The TRC solutions of these problems should be included in the {\tt assignment1.pdf} file and
the SQL solutions should be included in the {\tt assignment1.sql} file.

Here is an example of what is expected for your answers.

\begin{example}
Consider the constraint ``\emph{Each skill is the skill of a person.}''
In TRC, this constraint can be formulated as follows:
$$
\begin{array}{l}
\forall s\ Skill(s) \rightarrow \exists ps\, (personSkill(ps) \land ps.skill = s.skill)
\end{array}
$$
or, alternatively
$$
\begin{array}{l}
\lnot \exists s (Skill(s) \land \lnot \exists ps (personSkill(ps) \land ps.skill = s.skill)).
\end{array}
$$
This constraint can be specified using the following boolean SQL query.
\begin{verbatim}
select not exists (select 1
      	      	     from	  Skill	s
      	      	     where  not exists (select 1
      	      	      	      	          from   personSkill ps
      	      	      	      	          where  ps.skill = s.skill));
\end{verbatim}

\end{example}


\begin{enumerate}[resume]

\item  Each person works for a company and has at least two job skills.

\item  Some person has a salary that is strictly higher than the salary of each of his or her managers.


\item  Each employee and his or her managers work for the same company.


\end{enumerate}


\end{document}
